\bigskip

\definecolor{ffqqqq}{rgb}{1.,0.,0.}
\definecolor{qqqqff}{rgb}{0.,0.,1.}


\begin{tikzpicture}[line cap=round,line join=round,>=triangle 45,x=1.0cm,y=1.0cm]
\clip(3.4472431214179657,2.768260863550216) rectangle (20.370202046753114,12.44367887331736);
\draw [color=qqqqff] (7.053388964838967,6.992959893750945) circle (2.2631707245534325cm);
\draw (6.888401900107285,9.32) node[anchor=north west] {$n_{0}$};
\draw (7.595489320385912,9.13) node[anchor=north west] {$n_{1}$};
\draw (5.851340350365298,9.13) node[anchor=north west] {$n_{I-1}$};
\draw [rotate around={88.83496784682677:(6.98406480590215,6.994369701028559)}] (6.98406480590215,6.994369701028559) ellipse (3.8057416857828366cm and 3.060479680866765cm);
\draw (6.063466576448887,11.63616728691392) node[anchor=north west] {Cadran};
\draw [rotate around={88.83496784682187:(14.809165590318962,6.994369701028559)}] (14.809165590318962,7.300774249815967) ellipse (3.8057416857828437cm and 3.0604796808667727cm);
\draw (13.652871554106154,7.3571769237907625) node[anchor=north west] {$[n_{0}, n_{1}, \ldots,n_{I-1}]$};
\draw [->,line width=1.6pt,color=ffqqqq] (9.314597685794682,7.0871769237907625) -- (13.535023650726378,7.0871769237907625);
\draw (14.124263167625239,11.919002255025372) node[anchor=north west] {int[]};
\end{tikzpicture}

\bigskip

Nous représentons donc le cadran comme un tableau d'entiers de taille n trié par ordre croissant. 