Ce rapport a pour but de construire un algorithme correct qui prend en entrée un \textbf{cadran $C_{1}$} et un \textbf{entier naturel $n$} et qui génère le cadran résultant de la rotation de longueur $n$ de $C_{1}$.

Nous commencerons par étudier la théorie du problème, ensuite nous donnerons l'algorithme permettant de résoudre ce problème. Pour chaque partie de cet algorithme nous  ferons une correction partielle, ainsi qu'une preuve de terminaison. \\

Pour résoudre ce problème, nous avons décider d'utiliser deux sous-problèmes. Nous commencerons par présenter le problème général, en supposant le premier sous-problème existant. Puis nous passerons à la présentation de ce sous-problème, et nous finirons par le sous-problème le plus simple.  \\

Vous pouvez trouver le code Java complet de ce programme, en annexe de ce document. \\