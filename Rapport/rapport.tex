\documentclass[11pt,a4paper]{article}
\usepackage[utf8x]{inputenc}
\usepackage[french]{babel}
\usepackage[T1]{fontenc}
\usepackage{amsmath}
\usepackage[a4paper]{geometry}
\usepackage{amssymb}
\usepackage{pgf,tikz}
\usepackage{mathrsfs}
\usetikzlibrary{arrows}
\geometry{hscale=0.7,vscale=0.75,centering}

\author{\textit{Groupe 4}\\
Aymeric De Cocq, Solaiman El Jilali, Léonard Julémont, Florian Thuin}
\title{LINGI1122 - Méthodes de Conception de Programmes\\
Problème du cadran}
\date{11 Mai 2015}

\begin{document}
\maketitle

\section* {Introduction}
Ce rapport a pour but de construire un algorithme correct qui prend en entrée un \textbf{cadran $C_{1}$} et un \textbf{entier naturel $n$} et qui génère le cadran résultant de la rotation de longueur $n$ de $C_{1}$.

Nous commencerons par étudier la théorie du problème, ensuite nous donnerons l'algorithme permettant de résoudre ce problème. Pour chaque partie de cet algorithme nous  ferons une correction partielle, ainsi qu'une preuve de terminaison. \\

Pour résoudre ce problème, nous avons décider d'utiliser deux sous-problèmes. Nous commencerons par présenter le problème général, en supposant le premier sous-problème existant. Puis nous passerons à la présentation de ce sous-problème, et nous finirons par le sous-problème le plus simple.  \\

Vous pouvez trouver le code Java complet de ce programme, en annexe de ce document. \\

\section{Idée générale}
Pour que la lecture de ce rapport soit plus facile, commençons par décrire l'idée générale de la solution que nous proposons. \\

Le cadran peut être divisé en un certain nombre de parties ayant toutes le même nombre d'éléments, de telle manière à ce que le $X^{ieme}$ élément d'une partie doive être déplacé à la $X^{ieme}$ position d'une autre partie. Nous utilisons donc un sous-problème qui nous permet de déplacer l'élément de la position $X$ de chaque partie, à la $X^{ieme}$ position d'une autre partie, correspondant à un décalage de $N$ positions. L'algorithme principal fera, quand à lui, un appel au sous-problème, autant de fois qu'il y a d'éléments dans une partie. Cela permet donc de décaler tous les éléments d'un certain nombre de positions. 

 
\section{Théorie du problème}
Pour prouver la validité de notre programme, nous l'avons séparé en deux sous problèmes que nous aborderons séparément. La correction partielle et preuve de terminaison de ces deux sous-problèmes prouvera que le programme en entier est correct.\\
Dans le problème du cadran, il faut montrer que toutes les positions du tableau initiales seront déplacées de la valeur du déplacement. Notons le déplacement $A$ et la longueur $B$ du tableau donné en input. Nous allons aborder ce problème en deux sous problèmes: si A et B sont premiers entre eux et s'ils ne le sont pas.\\  

\subsection{A et B premiers entre eux}
\subsubsection* {Définition du problème}
Soient $A$ et $B$, deux nombres entiers positifs non nuls, premiers entre eux, avec $A < B$.\\
Les éléments x$_{i}$ de la suite x$_{i} = A \times i\ mod\ B$ sont uniques pour tous $i$ appartenant à $[0, B-1]$.
La suite des x$_{i}$ contiendra tous les éléments entre $[0, b-1]$ car ils sont uniques et il y en a $B$ (propriété du modulo : $0\leq x_i < B)$.

\subsubsection* {Preuve}
Prouvons que si $X$ et $Y$ sont deux entiers dans $[0, B-1]$ et que $A\times X\ mod\ B = A\times Y\ mod\ B$ alors $X = Y$. Autrement dit, deux éléments de la suite des x$_{i}$ ne sont égaux que s'il s'agit du même $i$, et donc du même élément. Nous notons la congruence modulo $N$ de deux nombres entiers positifs $A$ et $B$: $A \equiv B\ (mod\ N)$. Donc nous notons: $A\times X ≡ A\times Y\ (mod\ B)$.\\

Grâce aux propriétés du modulo, nous savons que $A\times X ≡ A\times Y\ (mod\ B) \Rightarrow A\times (X-Y) ≡ 0\ (mod\ B).$ Comme $A$ et $B$ sont premiers entre eux, on sait que leur PGCD vaut 1, puisqu'ils n'ont pas de diviseur commun autre que 1. Et par définition du PPCM : $PPCM(A,B) = \frac{A\times B}{PGCD(A,B)} = \frac{A\times B} {1} = A\times B$. Ce qui signifie que $B$ ne peut pas être un multiple de $A$.\\

Ceci implique que: $\neg  \exists k \in Z \wedge k>0\ :\ A\times k = B$. Si c'était le cas, $A\times k$ serait le PPCM de $A$ et $B$ ce qui n'est pas correct. Donc nous savons que : $A\times (X-Y) ≡ 0\ (mod\ B) \Rightarrow (X-Y) = 0$. C'est la seule possibilité pour que la relation de congruence soit respectée.\\

En effet, comme $X$ et $Y$ $\in$ $[0, B-1]$, $(X-Y)$ $\in$ $[-(B-1),B-1]$. Et comme $B$ n'est pas un multiple de $A$, $A$ "n'apporte rien" à la congruence avec 0. B doit donc être multiple de $(X-Y)$. Mais au vu des valeurs que peut prendre $(X-Y)$, la seule possibilité est que $(X-Y) = 0$.\\
On a donc que $X = Y$. Nous avons donc prouvé ce que nous voulions. \\

Intuitivement : si $A$ et $B$ sont premiers entre eux, cela signifie que faire $A\times i\ mod\ B$ avec $i$ dans $[0, B-1]$, nous donnera toujours un nombre différent. Donc pour notre problème, en partant à l'indice 0, on peut ajouter le décalage et avoir l'indice d'une case qui n'a pas encore été visitée. On peut ajouter une nouvelle fois le décalage et tomber sur une autre case non visitée. On peut faire cela $B$ fois. Grâce à ce nous avons montré, on sait que l'on ne tombera pas deux fois sur la même case et qu'on les parcourra toutes. 

\subsection*{A et B non premiers entre eux}
\subsubsection*{Preuve}
Soient $A$ et $B$ deux nombres entiers positifs avec $A < B$. Les éléments de la suite x$_{i,c} = (c + (A\times i))\ mod\ B$ sont uniques pour tous c appartenant à $[0, PGCD(A,B)-1]$ et $i$ appartenant à $[0, (\frac{B}{PGCD(A,B)})-1]$, et cette suite contiendra tous les éléments de $[0, B-1]$. \\

On sait que $PPCM(A,B) = \frac{A\times B}{PGCD(A,B)}$. Prenons $j = \frac{B}{PGCD(A,B)}$ donc $PPCM(A,B) = A\times J$.\\

Si l'on prend x$_{i} = A\times i\ mod\ B$ pour tout $i$ appartenant à $[0, J-1]$, on sait que les éléments x$_{i}$ seront uniques. En prenant $X$ et $Y$ appartenant à $[0, J-1]$, on sait que $A\times (X-Y) ≡ 0\ (mod\ B)$ ne sera vrai que pour $X=Y$. Etant donné que $(X-Y)$ appartient à $[-J+1, J-1]$. Comme nous l'avons dis plus haut, la seule possibilité que la congruence soit respectée est que $(X-Y) = 0$, donc $X=Y$. \\
Ainsi $A\times i\ mod\ B = x_{i}$, les x$_{i}$ seront uniques pour $i$ appartenant a $[0; J-1]$.\\

Maintenant, il ne manque plus qu'a ajouter le c dans cette expression pour arriver à l'expression énoncée plus haut. Les propriétés du modulo nous donne : si $e ≡ f\ (mod\ N)$ alors $e+g ≡ f+g\ (mod\ N)$. On peut donc écrire : $(c +(A\times i))\ mod\ B = x{i,c}$ pour $i$ dans $[0, J-1]$ et c dans $[0, PGCD(A,B) -1]$. Par le même raisonnement que celui que nous avons présenté, nous pouvons dire que les éléments $x{i,c}$ seront uniques et il y en aura B.\\
En effet, nous savons que $i$ peut prendre $\frac{B}{PGCD(A,B)}$ valeurs différentes et c peut prendre $PGCD(A,B)$ valeur différentes. Donc $x{i,c}$ peut prendre $B$ valeurs différents.\\

Intuitivement :On découpe le cadran en $\frac{B}{PGCD(A,B)}$ parties de PGCD éléments. 
$A\times i\ mod\ B$ nous permet d'atteindre le premier élément de chaque partie et c nous permet d'atteindre le c$^{ième}$ élément d'une partie. 
Cela nous permet de déplacer le c$^{ième}$ élément de chaque partie à la c$^{ième}$ position de la partie suivante. ce qui correspond à un décalage de $A$ éléments. 

\section{Conventions de représentations}
\bigskip

\definecolor{ffqqqq}{rgb}{1.,0.,0.}
\definecolor{qqqqff}{rgb}{0.,0.,1.}


\begin{tikzpicture}[line cap=round,line join=round,>=triangle 45,x=1.0cm,y=1.0cm]
\clip(3.4472431214179657,2.768260863550216) rectangle (20.370202046753114,12.44367887331736);
\draw [color=qqqqff] (7.053388964838967,6.992959893750945) circle (2.2631707245534325cm);
\draw (6.888401900107285,9.32) node[anchor=north west] {$n_{0}$};
\draw (7.595489320385912,9.13) node[anchor=north west] {$n_{1}$};
\draw (5.851340350365298,9.13) node[anchor=north west] {$n_{I-1}$};
\draw [rotate around={88.83496784682677:(6.98406480590215,6.994369701028559)}] (6.98406480590215,6.994369701028559) ellipse (3.8057416857828366cm and 3.060479680866765cm);
\draw (6.063466576448887,11.63616728691392) node[anchor=north west] {Cadran};
\draw [rotate around={88.83496784682187:(14.809165590318962,6.994369701028559)}] (14.809165590318962,7.300774249815967) ellipse (3.8057416857828437cm and 3.0604796808667727cm);
\draw (13.652871554106154,7.3571769237907625) node[anchor=north west] {$[n_{0}, n_{1}, \ldots,n_{I-1}]$};
\draw [->,line width=1.6pt,color=ffqqqq] (9.314597685794682,7.0871769237907625) -- (13.535023650726378,7.0871769237907625);
\draw (14.124263167625239,11.919002255025372) node[anchor=north west] {int[]};
\end{tikzpicture}

\bigskip

\section{Résolution du problème}

%\subsection{Problème général}

Les Pre, Post, Inv, H sont bien des expressions logiques

\subsection{Sous-problème 1}

\subsubsection{Spécifications}

\paragraph{Pré :}

\begin{tabular}{ll}
  & \textit{tab} est un tableau d'entiers \\
  $\dot{\wedge}$ & tab != null \\
  $\dot{\wedge}$ & j est un indice tel que $0 \le j < \mathrm{tab.length}$
\end{tabular}

\paragraph{Post :}

Le tableau est modifié de manière que \textit{tmp} a été placé à l'indice $j$ du tableau \textit{tab}.

\paragraph{Résultat :}

La valeur présente à l'indice $j$ de \textit{tab} lors de l'appel à la méthode est renvoyée.

\paragraph{ \{Pré\} Instructions \{Post\} }

\{tab = $tab_{0}$, j = $j_{0}$, tmp = $tmp_{0}$\}
\{$tmp2_{1}$ = tab[$j_{0}$], $tab_{0}$[j] = $tmp_{0}$\}

L'exécution symbolique prouve que les instructions permettent d'arriver à la postcondition lorsque la précondition est respectée. Le résultat renvoyé est la valeur dans tmp2 qui contient bien la valeur prévue par le résultat.

\subsection{Sous-problème 2}

\subsubsection*{Specifications}
\noindent \textbf{En tête :} public static void switchCadran2(int[] tab, int startIndex, int decalage, int nSwitch)\\
\noindent \textbf{Pré : } tab est un tableau d'entiers\\
 	  \indent \indent $\dot{\wedge}$ tab != null\\
	  \indent \indent $\dot{\wedge}$ startIndex est un indice tel que 0 $\leq$ startIndex < tab.length \\
	  \indent \indent $\dot{\wedge}$ decalage est un entier positif \\
	  \indent \indent $\dot{\wedge}$ nSwitch est un entier positif 0 $\leq$ nSwitch < tab.length\\
\textbf{Post :}  le tableau est modifié de manière à ce que les éléments aux positions P$_{i}$ sont déplacés aux positions P$_{i+1}$ pour tout $i$ appartenant à $[0,nSwitch[$ tel que $P_{x} = (startIndex + (decalage\times x))\ mod\ tab.length$\\
\textbf{Résultat :}  - \\
\textbf{Environement: } int switchDone, int i, int tmp\\

\noindent \textbf{Invariant:}\\
\noindent startIndex = startIndex$_{0}$\\
$\wedge$ $0 \leq$ startIndex < tab.length \\
$\wedge$ decalage = decalage$_{0}$\\
$\wedge$ nswitch = nswitch$_{0}$\\
$\wedge$ $0\leq switchDone \leq nSwitch < tab.length$\\
$\wedge$ $0\leq i < tab.length$\\
$\wedge$ $i = startindex + ((switchDone+1)\times decalage) \% tab.length$\\
$\wedge$ $tmp = tab[(startIndex + (switchDone\times decalage))  \% tab.length]$\\
$\wedge$ Les éléments x$_{j}$ tels que $\forall j$, $j$ dans $Z$, $j<switchDone$ on a $x_{j}=tab_{0}[(startindex + (j\times decalage))\ \% tab.length]$ sont déplacés aux index $(startindex + ((j+1)* decalage))\%tab.length$\\

\noindent \textbf{Condition d'arrêt:} switchDone == nSwitch\\

\noindent \textbf{Init:}\\
switchDone := 0;\\
i := (startIndex + decalage) \% tab.length;\\
tmp := tab[startIndex];\\

\noindent \textbf{Iter: }\\
tmp := switchUnique(tab, i, tmp);\\
i := (i+decalage) \% tab.length;\\
switchDone := switchDone + 1;\\

\noindent \textbf{Clot: } -\\



\subsubsection*{$\{$Pre$\}$ Init $\{$Inv$\}$}

$\{startIndex = startIndex_{0}, decalage = decalage_{0}, nswitch = nswitch_{0} \}$ tq (startIndex$_{1}$ = startIndex$_{0}$ \\
$\wedge$ decalage$_{1}$ = decalage$_{0}$\\
$\wedge$ nswitch$_{1}$ = nswitch$_{0}$\\
$\wedge$ tab = tab$_{0}$ [val$_{1}^{0}$, val$_{1}^{1}$, ... ,val$_{1}^{tab.length-1}$])\\

INIT \\

$\{switchDone = 0,\ i = (startIndex_0 + decalage_0)\ \% tab.length, tmp = tab_0[startindex_0]\}$ tq (...)\\
on a bien : 
\begin{itemize}
	\item $0\leq switchDone < tab.length$
	\item $i = startindex + (switchDone+1\times decalage)\ \% tab.length $car$ switchDone = 0$
	\item $tmp = tab[(startIndex + (SwitchDone\times decalage))\ \% tab.length]$ car $switchDone = 0$
	\item il n'existe pas de $j \in \mathbb{Z}$, $j$< SwitchDone = 0 donc la dernière conjonction de notre invariant est vraie. \\
\end{itemize}


L'invariant est donc vérifié sur Init.

\subsubsection*{$\{$Inv $\&\&$ !H$\}$ Iter $\{$Inv$\}$}

$\{ switchDone = switchDone_{1}, i = i_{1}, tmp = tmp_{1}, startIndex = startIndex_{1}, decalage = decalage_{1}, nswitch = nswitch_{1}, tab = tab_{1} \}$ tq (startIndex$_{1}$ = startIndex$_{0}$\\
$\wedge$ $0 \leq$ startIndex < tab.length \\
$\wedge$ decalage$_{1}$ = decalage$_{0}$\\
$\wedge$ nswitch$_{1}$ = nswitch$_{0}$\\
$\wedge$ $0\leq switchDone_{1} \leq nSwitch_{1} < tab.length$\\
$\wedge$ $0\leq i_{1} < tab.length$\\
$\wedge$ $i_{1} = startIndex_{1} + (switchDone_{1}+1\times decalage_{1}) \% tab.length$\\
$\wedge$ $tmp_{1} = tab[(startIndex_{1} + (switchDone_{1}\times decalage_{1}))  \% tab.length]$\\
$\wedge$ Les éléments x$_{j}$ tels que $\forall j$, $j$ dans $Z$, $j<switchDone_{1}$ on a $x_{j}=tab_{0}[(startindex_{1} + (j\times decalage_{1}))\ \% tab.length]$ sont déplacés aux index $(startindex_{1} + ((j+1)* decalage_{1}))\%tab.length$\\

La fonction $switchUnique$ peut être appelé car les pré-conditions sont biens respectées. En effet, tab est bien un tableau d'entiers, tab est non null et $i$ est un indice valide du tableau. \\

\begin{center}
$tmp = switchUnique(tab, i, tmp);$\\
\end{center}

Donc $tmp = tab_{1}[i_1]$, ce qui correspond à $tmp = tab_{1}[startindex_{1} + ((switchDone_{1}+1)\times decalage_{1}) \% tab.length]$.\\

Le tableau a été modifié : $tab = tab_{1}[val_{1}^{0},val_{1}^{1}, ..., val_{1}^{(i_{1}-1)}, tmp_{1}, ...,val_{1}^{tab.length-1}]$.\\

\begin{center}
  $i = (i+decalage) \% tab.length;$\\
 \end{center} 
 
 Dans l'environnement : $i = i_{1} + decalage \% tab.length = startindex_{1} + ((switchDone_{1}+1)\times decalage_{1}) \% tab.length  + decalage_{1} \%tab.length = startIndex_{1} + ((swhitchDone_{1}+2)\times decalage_{1}) \%tab.length$.\\
 
\begin{center}
 $switchDone++;$\\
\end{center}
 
Après l'ITER, l'environnement devient donc : \\

\noindent$\{switchDone = switchDone_{1}+1,\ i = startIndex_{1} + ((swhitchDone_{1}+2)\times decalage_1) \%\ tab.length,$\\
$\ tmp = tab_{1}[startindex_{1}$ $+ ((switchDone_{1}+1)\times decalage_{1}) \% tab.length],\ startIndex = startIndex_{1}, decalage = decalage_{1},\ nswitch = nswitch_{1},\ tab = tab_{1}[val_{1}^{0},val_{1}^{1}, ..., val_{1}^{(i_{1}-1)}, tmp_{1}, ...,val_{1}^{tab.length-1}]\}$\\
tq (...).\\

On obtient bien notre invariant:
\begin{itemize}
	\item comme switchDone est différent de nSwitch et inférieur ou égal, on sait qu'il est toujours inférieur à tab.length
	\item $tmp = tab_{1}[startindex_{1} + ((switchDone_{1}+1)\times decalage_{1}) \% tab.length]$ 
		\subitem $= tab_{1}[startindex + (switchDone\times decalage) \% tab.length]$
	\item $i = startIndex_{1} + ((swhitchDone_{1}+2)\times decalage_{1}) \% tab.length $
		\subitem $= startIndex + ((swhitchDone+1)\times decalage) \%tab.length$
	\item pour $tab$, on a $tab = tab_{1}[val_{1}^{0},val_{1}^{1}, ..., val_{1}^{((i_{1})-1)}, tmp_{1}, ...]$, avec $tmp_{1} = tab_{1}[(startIndex_{1} + (SwitchDone_{1}\times decalage_{1})) \% tab.length]$, et donc avec le nouvel environnement : $tmp_{1} = tab_{1}[(startIndex + ((switchDone-1)\times decalage)) \% tab.length].$ 
	\item Grâce à notre invariant nous savons déjà que les éléments $x_{j}$ tels que $\forall j$, $j \in \mathbb{Z}$, $j<switchDone_{1}$ on a $x_{j}=tab_{0}[(startindex_{1} + (j\times decalage_{1})) \% tab.length]$ sont déplacés aux index $(startindex_{1} + ((j+1)\times decalage_{1}))\% tab.length$. On a maintenant $\forall j, j$ dans $\mathbb{Z}$ et $j<switchDone=switchDone_{1} + 1$, donc seul l'élément $x_{j}$ avec $j= switchDone - 1 = switchDone_{1}$ a du être déplacé. Et c'est le cas, il s'agit de l'élément qui était dans $tmp_{1}$ et qui a été placé à l'index :$ startindex_{1} + ((switchDone_{1}+1)\times decalage_{1}) \% tab.length$. Grâce à notre théorie du problème nous savons que cet élément n'est pas un élément qui avait déjà été déplacé. Ceci nous donne donc la dernière conjonction de notre invariant. \\
\end{itemize}

Notre invariant est donc vérifié sur Iter. \\

\subsubsection*{$\{$Inv $\&\&$ H$\}$ Clot $\{$Post$\}$}

$\{switchDone = switchDone_{1}, i = i_{1}, tmp = tmp_{1}, startIndex = startIndex_{1}, decalage = decalage_{1}, nswitch = nswitch_{1}, tab = tab_{1}\}$\\ 
tq $(startIndex_{1} = startIndex_{0}$\\
$\wedge$ $0 \leq startIndex < tab.length$\\
$\wedge$ $decalage_{1} = decalage_{0}$\\
$\wedge$ $nSwitch_{1} = nSwitch_{0}$\\
$\wedge$ $0 \leq switchDone_{1} \leq nSwitch_{1} < tab.length$\\
$\wedge$ $0\leq switchDone_{1} < tab.length$\\
$\wedge$ $0\leq i_{1} < tab.length$\\
$\wedge$ $i_{1} = startindex_{1} + ((switchDone_{1}+1)\times decalage_{1}) \% tab.length$\\
$\wedge$ $tmp_{1} = tab_{1}[(startIndex_{1} + (switchDone_{1}\times decalage_{1}))  \% tab.length]$\\
$\wedge$ les éléments $x_{j}$ tels que $\forall j, j$ dans $\mathbb{Z}, j<switchDone_{1}$ on a $x_{j}=tab_{0}[(startindex_{1} + (j\times decalage_{1})) \% tab.length]$ sont déplacés aux index ($startindex_{1} + ((j+1)\times  decalage_{1}))\% tab.length$\\
$\wedge$ $switchDone_{1} == nSwitch_{1})$ \\

La post condition de notre sous problème demande que : le tableau est modifié de manière à ce que les éléments aux positions $P_{i}$ sont déplacés aux positions $P_{i+1}$ pour tout $i$ appartenant à $[0,nSwitch[$ tel que $P_{x} = (startIndex + (decalage\times x)) \% tab.length$.

Comme H est vrai, nous avons $switchDone == nSwitch$. Donc en remplaçant switchDone par nSwitch dans la dernière conjonction de notre invariant, nous savons que : les éléments $x_{j}$ tels que $\forall j, j$ dans $\mathbb{Z}, j<nSwitch$ on a $x_{j}=tab_{0}[(startindex + (j\times decalage)) \% tab.length]$ sont déplacés aux index $(startindex + ((j+1)\times decalage)) \% tab.length.$ 
Cela correspond exactement à notre postcondition.\\

Nous avons donc vérifié le tripler d'Hoar-Mana, ce qui nous montre que cet invariant est correct pour cet algorithme. 
Nous avons donc effectué la correction partielle de ce sous problème. 


\subsubsection*{Preuve de terminaison}
Variant  : $tab.length - switchDone$

Inv $\Rightarrow 0 \leq switchDone \leq tab.length$ donc $tab.length - switchDone \geq 0$. Notre variant est bien positif ou nul.\\ 

Vérifions maintenant que notre variant décroit sur Iter.\\

Prenons Inv $\&\&$ !H \\

E = $\{ switchDone = switchDone_{1}, i = i_{1}, tmp = tmp_{1}, startIndex = startIndex_{1}, decalage = decalage_{1}, nswitch = nswitch_{1}, tab = tab_{1}\}$\\
tq $(startIndex_{1} = startIndex_{0}$
$\wedge$ $0 \leq startIndex < tab.length$\\
$\wedge$ $decalage_{1} = decalage_{0}$\\
$\wedge$ $nSwitch_{1} = nSwitch_{0}$\\
$\wedge$ $0 \leq switchDone_{1} \leq nSwitch_{1} < tab.length$\\ 
$\wedge$ $0 \leq switchDone_{1} < tab.length$\\
$\wedge$ $0 \leq i_{1} < tab.length$\\
$\wedge$ $i_{1} = startindex_{1} + ((switchDone_{1}+1)\times decalage_{1}) \% tab.length$
$\wedge$ $tmp_{1} = tab_{1}[(startIndex_{1} + (switchDone_{1}\times decalage_{1})) \% tab.length]$
$\wedge$ les éléments $x_{j}$ tels que $\forall j, j$ dans $\mathbb{Z}$, $j<switchDone_{1}$ on a $x_{j}=tab_{0}[(startindex_{1} + (j\times decalage_{1})) \% tab.length]$ sont déplacés aux index $(startindex_{1} + ((j+1)\times decalage_{1})) \% tab.length$.
$\wedge$ $switchDone_{1} != nSwitch_{1})$.\\

La seule instruction intéressante est : switchDone++.\\
Cela nous donne un nouvel environnement :\\

E' = $\{switchDone = switchDone_{1}+1, ...\}$ tq (...).\\

$Var(E) = tab.length - switchDone_{1}$
$Var(E') = tab.length - switchDone = tab.length - (switchDone_{1}+1)$

On a donc : $Var(E) = tab.length - switchDone_{1} > tab.length - (switchDone_1 +1) = tab.length - switchDone = Var(E')$.\\

On a donc bien un variant qui décroit sur Iter.\\

Ceci nous montre que ce sous problème ce finira toujours. 

Nous avons donc fini la correction partielle et prouvé que cet algorithme s'arrête toujours. On peut donc en conclure que cet algorithme est correct, suivant ces spécifications. \\






\subsection{Sous-problème 2}

Invariant :

\begin{tabular}{ll}

& $\mathrm{decalage}=\mathrm{decalage}_{0}$ \\
$\dot{\wedge}$ & $\forall i \in \mathbb{Z} \mid i < \mathrm{switchDone} $ \\
$\dot{\wedge}$ & $\forall j \in \mathbb{Z} \mid j < \frac{\mathrm{tab.length}}{\mathrm{PGCD}}$ \\
$\Rightarrow$ & les éléments aux indices $tab_{0}[(j+i) \% tab.length]$ sont déplacés aux indices $((j\times decalage)+i) \% tab.length$ \\
\end{tabular}

\subsubsection*{\{Pre\} Init \{Inv\}}

$\{decalage=decalage_{0}, PGCD =\cdots, switchDone=0$
$PGCD_{1} = \cdots$

On a bien qu'il n'existe aucun $i < \mathrm{switchDone}$ et donc
qu'aucun élément n'est déplacé au sein de tab.

L'invariant est donc vérifié sur INIT.

\subsubsection*{\{Inv \&\& !H \} Iter \{Inv\}}

A chaque itération, switchCadran2 est appelé avec les arguments : tab,
switchDone (incrémenté), décalage et tab.length / PGCD. De ce fait,
selon la spécification de switchCadran2 prouvé précédemment, nous avons
que chaque élément $x \in [0\cdots
    \frac{\mathrm{tab.length}}{\mathrm{PGCD}}[$ sont déplacés à la
        position $(x\times \mathrm{decal})+i \% \mathrm{tab.length}$ avec $i < \mathrm{switchDone}$

On a donc que les éléments aux indices
$tab_{0}[(j+i)\%\mathrm{tab.length}]$ sont déplacés aux indices
$((j\times decalage) + i) \% \mathrm{tab.length}$ pour chaque $i <
\mathrm{switchDone}$
et $j \leq \frac{\mathrm{tab.length}}{\mathrm{PGCD}}$

\subsubsection*{\{Inv \&\& H\} Clot \{Post\}}

 Nous avons que switchDone=PGCD ce qui implique qu'on a déplacé
$\mathrm{PGCD}\times \frac{\mathrm{tab.length}}{\mathrm{PGCD}}$ éléments et
que chacun d'entre eux est déplacé une et une seule fois (comme démontré
plus haut). Tous les élémetnts de tab sont donc déplacés de décal positions.


\section{Exécution Symbolique}

\section{Variants}


\end{document}
