Dans le problème du cadran, il faut montrer que toutes les positions du tableau initiales seront déplacées de la valeur du déplacement. Notons le déplacement $A$ et la longueur $B$ du tableau donné en input.\\  

Comme expliqué dans la section ci-dessus, nous voulons décomposer le cadran en différentes parties ayant la même taille. Dans cette partie nous allons montrer que nous pouvons déplacer les éléments de A positions, comme expliqué ci-dessus, sans repasser sur une case déjà visitée. 

\subsubsection*{Théorème}
Soient $A$ et $B$ deux nombres entiers positifs avec $A < B$. Les éléments de la suite x$_{i,c} = (c + (A\times i))\ mod\ B$ sont uniques pour tous $c$ appartenant à $[0, PGCD(A,B)-1]$ et $i$ appartenant à $[0, (\frac{B}{PGCD(A,B)})-1]$, et cette suite contiendra tous les éléments de $[0, B-1]$. 

\subsubsection*{Preuve}

On sait que $PPCM(A,B) = \frac{A\times B}{PGCD(A,B)}$. Prenons $J = \frac{B}{PGCD(A,B)}$ donc $PPCM(A,B) = A\times J$.\\

Si l'on prend x$_{i} = A\times i\ mod\ B$ pour tout $i$ appartenant à $[0, J-1]$, on sait que les éléments x$_{i}$ seront uniques. En prenant $X$ et $Y$ appartenant à $[0, J-1]$, on sait que $A\times (X-Y) ≡ 0\ (mod\ B)$ ne sera vrai que pour $X=Y$. En effet, étant donné que $(X-Y)$ appartient à $[-J+1, J-1]$, donc à $[- \frac{B}{PGCD(A,B)} +1,\frac{B}{PGCD(A,B)} -1] $, on a $A\times (X-Y)$ appartient à $[- PPCM(A,B) +1, PPCM(A,B) )-1]$. On voit donc que $A\times (X-Y)$ n'atteint jamais le plus petit commun multiple de A et B.\\
La seule possibilité pour que la congruence soit respectée est donc que $(X-Y) = 0$, donc $X=Y$. Ainsi $A\times i\ mod\ B = x_{i}$, les x$_{i}$ seront uniques pour $i$ appartenant à $[0, J-1]$.\\

Maintenant, il ne manque plus qu'a ajouter le $c$ dans cette expression pour arriver à l'expression énoncée plus haut. Les propriétés du modulo nous donne : si $e ≡ f\ (mod\ N)$ alors $e+g ≡ f+g\ (mod\ N)$. On peut donc écrire : $(c +(A\times i))\ mod\ B = x_{i,c}$ pour $i$ dans $[0, J-1]$ et c dans $[0, PGCD(A,B) -1]$. Par le même raisonnement que celui que nous avons présenté, nous pouvons dire que les éléments $x_{i,c}$ seront uniques et il y en aura B.\\
En effet, nous savons que $i$ peut prendre $\frac{B}{PGCD(A,B)}$ valeurs différentes et $c$ peut prendre $PGCD(A,B)$ valeurs différentes. Donc $x_{i,c}$ peut prendre $B$ valeurs différents.\\

Intuitivement : On découpe le cadran en $\frac{B}{PGCD(A,B)}$ parties de $PGCD(A,B)$ éléments. 
$A\times i\ mod\ B$ nous permet d'atteindre le premier élément de la $i^{ieme}$ partie et $c$ nous permet d'atteindre le $c^{ieme}$ élément d'une partie. 
Cela nous permet de déplacer le $c^{ieme}$ élément de chaque partie à la $c^{ieme}$ position de la partie correspondant à un décalage de $A$ éléments.