Pour que la lecture de ce rapport soit plus facile, commençons par décrire l'idée générale de la solution que nous proposons. \\

Le cadran peut être divisé en un certain nombre de parties ayant toutes le même nombre d'éléments, de telle manière à ce que le $X^{ieme}$ élément de la première partie doive être déplacé à la $X^{ieme}$ position de la partie suivante. Nous utilisons donc un sous-problème qui nous permet de déplacer l'élément de la position $X$ de chaque partie, à la $X^{ieme}$ position de la partie suivante. L'algorithme principal fera, quand à lui, un appel au sous-problème, autant de fois qu'il y a d'éléments dans une partie. Cela permet donc de décaler tous les éléments d'un certain nombre de positions. \\