Pour que la lecture de ce rapport soit plus facile, nous allons décrire ici l'idée générale de la solution que nous proposons. \\

Le cadran peut être divisé en un certain nombres de parties ayant toutes le même nombres d'éléments, de telle manière à ce que le premier élément de la première partie doive être déplacé à la première position de la partie suivante. Nous utilisons donc un sous-problème qui nous permet de déplacé le $X^{ieme}$ élément de la $Y^{ieme}$ partie, à la position $X$ de la partie $Y^{ieme}$ modulo la taille du tableau. L'algorithme principal fera, quand à lui, un appel au sous-problème autant de fois qu'il y a d'élément dans une partie. Cela permet décaler les éléments à toutes les positions dans les parties, dans chaque partie. 