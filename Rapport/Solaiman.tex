\subsection{Problème principal}
\subsubsection*{Spécifications}
\noindent \textbf{En-tête :} public static void switchCadran(int[] tab, int decalage)\\
\noindent \textbf{Pré :} tab est un tableau d'entiers != null\\
\indent $\dot{\wedge}$ decalage est un nombre entiers positifs\\
\textbf{Post :} tous les éléments du tableau aux positions i sont déplacés aux positions (i+decalage \% tab.length)\\
\textbf{Résultat :} - \\

\noindent \textbf{Environement: }\\
int switchDone\\
int pgcd\\

\noindent \textbf{Invariant :}

\begin{tabular}{lp{14cm}}

& $\mathrm{decalage}=\mathrm{decalage}_{0}$ \\
$\dot{\wedge}$ & $\forall i \in \mathbb{Z} \mid i < \mathrm{switchDone} $, $\forall j \in \mathbb{Z} \mid j < \frac{\mathrm{tab.length}}{\mathrm{PGCD}}$ \\
$\Rightarrow$ & les éléments aux indices $tab_{0}[(j \times decalage + i) \% tab.length]$ sont déplacés aux indices $(((j+1)\times decalage)+i) \% tab.length$ \\
\end{tabular}\\

\noindent \textbf{Condition d'arrêt: }\\
switchDone==pgcd \\

\noindent \begin{tabular}{l|l}
\textbf{Init: } & \textbf{Iter: }\\
switchDone := 0; & switchCadran2(tab, switchDone, decalage, tab.length / pgcd); \\
pgcd := pgcd(tab.length,decalage); & switchDone=switchDone+1;\\
\end{tabular}\\

\noindent \textbf{Clot: } -

\subsubsection*{\{Pre\} Init \{Inv\}}

\{decalage = $decalage_{0}$ \}
$\wedge$ tab = tab$_{0}$ [val$_{1}^{0}$, val$_{1}^{1}$, ... ,val$_{1}^{tab.length-1}$])\\

INIT

\{switchDone = 0, pgcd = PGCD(tab.length, decalage)\} \footnote{Par manque d'espace nous ne prouverons pas l'exactitude de la fonction PGCD mais elle peut être prouvée de la même manière que vue au cours}

On a bien qu'il n'existe aucun $i < \mathrm{switchDone}$, aucun élément n'est déplacé au sein de tab.

L'invariant est donc vérifié sur INIT.

\subsubsection*{\{Inv \&\& !H \} Iter \{Inv\}}

A chaque itération, switchCadran2 est appelé avec les arguments : tab,
switchDone (incrémenté), décalage et tab.length / PGCD. De ce fait,
selon la spécification de switchCadran2 prouvé précédemment, nous avons
que chaque élément $x \in [0\cdots
    \frac{\mathrm{tab.length}}{\mathrm{PGCD}}[$ est déplacé à la
        position $(x\times \mathrm{decal})+i \% \mathrm{tab.length}$ avec $i < \mathrm{switchDone}$

On a donc que les éléments aux indices $tab_{0}[(j \times decalage + i) \% tab.length]$ sont déplacés aux indices $(((j+1)\times decalage)+i) \% tab.length$ pour chaque $i < \mathrm{switchDone}$ et $j \leq \frac{\mathrm{tab.length}}{\mathrm{PGCD}}$ 

\subsubsection*{\{Inv \&\& H\} Clot \{Post\}}

 Nous avons que switchDone=PGCD ce qui implique qu'on a déplacé
$\mathrm{PGCD}\times \frac{\mathrm{tab.length}}{\mathrm{PGCD}}$ éléments et
que chacun d'entre eux est déplacé une et une seule fois (comme démontré
plus haut). Tous les élémetnts de tab sont donc déplacés de décal positions.

\subsubsection*{Preuve de terminaison}
TODO: Preuve de terminaison
%TODO Preuve de terminaison