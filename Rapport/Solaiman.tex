\subsection{Problème principal}
\subsubsection*{Spécifications}
\noindent \textbf{En-tête :} public static void switchCadran(int[] tab, int decalage)\\
\noindent \textbf{Pré :} tab est un tableau d'entiers != null\\
\indent $\dot{\wedge}$ decalage est un nombre entier positif\\
\textbf{Post :} tous les éléments du tableau aux positions i sont déplacés aux positions (i+decalage) \% tab.length\\
\textbf{Résultat :} - \\

\noindent \textbf{Environement: } int switchDone, int PGCD\\

\noindent \textbf{Invariant :}

\begin{tabular}{lp{14cm}}

& $\mathrm{decalage}=\mathrm{decalage}_{0}$ $\wedge 0 \leq switchDone \leq pgcd$\\
$\dot{\wedge}$ & $\forall i \in \mathbb{Z} \mid i < \mathrm{switchDone} $, $\forall j \in \mathbb{Z} \mid j < \frac{\mathrm{tab.length}}{\mathrm{PGCD}}$, les éléments $tab_{0}[((j \times \mathrm{decalage}) + i) \% \mathrm{tab.length}]$ sont déplacés aux indices $(((j+1)\times \mathrm{decalage})+i) \% \mathrm{tab.length}$ \\
\end{tabular}\\

\noindent \begin{tabular}{ll}
\textbf{Init: } & \textbf{Iter: }\\
switchDone := 0; & switchCadran2(tab, switchDone, decalage, tab.length / PGCD); \\
PGCD := PGCD(tab.length,decalage); & switchDone := switchDone+1;\\
 & \\
\textbf{H:}switchDone==PGCD & \textbf{Clot: } -\\
\end{tabular}\\

\subsubsection*{Preuve partielle de l'algorithme}
\textbf{\{Pre\} Init \{Inv\}}

\{decalage = decalage$_{0}$, tab = tab$_{0}$ [val$_{0}^{0}$, val$_{0}^{1}$, ... ,val$_{0}^{\mathrm{tab.length}-1}$])\} tq (tab est un tableau d'entiers != null
$\dot{\wedge}$ decalage est un nombre entier positif)\\

INIT

\{switchDone = 0, pgcd = PGCD(tab.length, decalage), ...\} \footnote{Par manque d'espace nous ne prouverons pas l'exactitude de la fonction PGCD mais elle peut être prouvée de la même manière que vue au cours} tq(Idem)\\

On a bien qu'il n'existe aucun $i \in \mathbb{Z} tel que i< \mathrm{switchDone} = 0$, donc la dernière conjonction de notre invariant respectée. \\

L'invariant est donc vérifié sur INIT.

\textbf{\{Inv \&\& !H \} Iter \{Inv\}}

\{decalage = $decalage_{1}$, tab = tab$_{1}$ [val$_{1}^{0}$, val$_{1}^{1}$, ... ,val$_{1}^{\mathrm{tab.length-1}}$]), switchDone = switchDone$_{1}$, pgcd = pgcd$_{1}$ \} tq (INV $\wedge$ switchDone $\neq$ pgcd )\\

ITER \\

L'appel à SwitchCadran2 peut être effectué car les préconditions sont biens respectées :
\begin{enumerate}
 \item tab donné à switchCadran2 est le tableau non nul composé d'entiers reçu en paramètre dont les éléments ont éventuellement été deplacés
 \item switchDone est positif
 \item decalage respecte les preconditions de switchCadran et est donc positif
 \item tab.length / PGCD qui est bien compris entre 0 et tab.length
\end{enumerate}


De ce fait, selon les spécifications de switchCadran2 prouvées précédemment, nous avons
que chaque élément à la position $P_{i}$  est déplacé à la position P$_{i+1}$ pour$ i \in [0\cdots
    \frac{\mathrm{tab.length}}{\mathrm{PGCD}}[$, avec $P_{x} = (x\times \mathrm{decalage})+switchDone \% \mathrm{tab.length}$\\

On a donc que les éléments aux indices $tab_{0}[((j \times \mathrm{decalage}) + i) \% \mathrm{tab.length}]$ sont déplacés aux indices $(((j+1)\times \mathrm{decalage})+i) \% \mathrm{tab.length}$ pour chaque $i < \mathrm{switchDone}$ et $j < \frac{\mathrm{tab.length}}{\mathrm{PGCD}}$.
A chaque itération, nous avons donc que $\mathrm{switchDone} \times \frac {\mathrm{tab.length}}{\mathrm{PGCD}}$ décalages ont été effectués\\

Notre invariant est bien respecté pour Iter.\\

\textbf{\{Inv \&\& H\} Clot \{Post\}}\\
\{decalage = $decalage_{1}$, tab = tab$_{1}$ [val$_{1}^{0}$, val$_{1}^{1}$, ... ,val$_{1}^{\mathrm{tab.length-1}}$]), switchDone = switchDone$_{1}$, pgcd = pgcd$_{1}$\} tq (INV $\wedge$ switchDone == pgcd )\\

 Nous avons que switchDone=PGCD ce qui implique qu'on a déplacé
$\mathrm{PGCD}\times \frac{\mathrm{tab.length}}{\mathrm{PGCD}}$ éléments et
que chacun d'entre eux est déplacé une et une seule fois (comme démontré lors de la théorie du problème). Tous les élémetnts de tab sont donc déplacés de \og decalage \fg{}  positions. 

\subsubsection*{Preuve de terminaison}

\textbf{Variant}  : pgcd - switchDone

INIT et INV:  $0 \leq \mathrm{switchDone} \leq$ pgcd sachant que pgcd ne renvoie que des valeurs positives donc $pgcd - \mathrm{switchDone} \geq 0$. Notre variant est bien positif ou nul.\\ 

Vérifions maintenant que notre variant décroit sur Iter. Prenons Inv $\&\&$ !H \\

E = $\{$ switchDone = switchDone$_{1}$, pgcd = pgcd$_{1}$, tab = tab$_{1}\}$ tq (INV $\wedge$ switchDone$_{1}$ $\neq$ pgcd$_{1})$.\\

Une seule instruction va nous intéresser ici : 

\begin{center}
  switchDone++;\\
\end{center} 
 
Cela nous donne un nouvel environnement :\\

E' = $\{$switchDone = switchDone$_{1}$+1, ...\} tq (Idem).\\

Var(E) = pgcd - switchDone$_{1}$
Var(E') = pgcd - switchDone = PGCD - (switchDone$_{1}+1)$

On a donc : Var(E) = pgcd - switchDone$_{1}$ > pgcd - (switchDone$_1$ +1) = pgcd - switchDone = Var(E').\\

On a donc bien un variant qui décroit sur Iter.\\

Ceci nous montre que le problème général se finira toujours. 

\subsubsection*{Conclusion}

Nous avons donc fini la correction partielle et prouvé que cet algorithme s'arrête toujours. On peut donc en conclure que cet algorithme est correct, suivant ses spécifications. 
