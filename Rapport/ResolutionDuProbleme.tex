%\subsection{Problème général}

Les Pre, Post, Inv, H sont bien des expressions logiques

\subsection{Sous-problème 1}

\subsubsection{Spécifications}

\paragraph{Pré :}

\begin{tabular}{ll}
  & \textit{tab} est un tableau d'entiers \\
  $\dot{\wedge}$ & tab != null \\
  $\dot{\wedge}$ & j est un indice tel que $0 \le j < \mathrm{tab.length}$
\end{tabular}

\paragraph{Post :}

Le tableau est modifié de manière que \textit{tmp} a été placé à l'indice $j$ du tableau \textit{tab}.

\paragraph{Résultat :}

La valeur présente à l'indice $j$ de \textit{tab} lors de l'appel à la méthode est renvoyée.

\paragraph{ \{Pré\} Instructions \{Post\} }

\{tab = $tab_{0}$, j = $j_{0}$, tmp = $tmp_{0}$\}
\{$tmp2_{1}$ = tab[$j_{0}$], $tab_{0}$[j] = $tmp_{0}$\}

\subsection{Sous-problème 2}

\subsubsection*{Specifications}
\noindent \textbf{En tête :} public static void switchCadran2(int[] tab, int startIndex, int decalage, int nSwitch)\\
\noindent \textbf{Pré : } tab est un tableau d'entiers\\
 	  \indent \indent $\dot{\wedge}$ tab != null\\
	  \indent \indent $\dot{\wedge}$ startIndex est un indice tel que 0 $\leq$ startIndex < tab.length \\
	  \indent \indent $\dot{\wedge}$ decalage est un entier positif \\
	  \indent \indent $\dot{\wedge}$ nSwitch est un entier positif 0 $\leq$ nSwitch < tab.length\\
\textbf{Post :}  le tableau est modifié de manière à ce que les éléments aux positions P$_{i}$ sont déplacés aux positions P$_{i+1}$ pour tout $i$ appartenant à $[0,nSwitch[$ tel que $P_{x} = (startIndex + (decalage\times x))\ mod\ tab.length$\\
\textbf{Résultat :}  - \\
\textbf{Environement: } int switchDone, int i, int tmp\\

\noindent \textbf{Invariant:}\\
\noindent startIndex = startIndex$_{0}$\\
$\wedge$ $0 \leq$ startIndex < tab.length \\
$\wedge$ decalage = decalage$_{0}$\\
$\wedge$ nswitch = nswitch$_{0}$\\
$\wedge$ $0\leq switchDone \leq nSwitch < tab.length$\\
$\wedge$ $0\leq i < tab.length$\\
$\wedge$ $i = startindex + ((switchDone+1)\times decalage) \% tab.length$\\
$\wedge$ $tmp = tab[(startIndex + (switchDone\times decalage))  \% tab.length]$\\
$\wedge$ Les éléments x$_{j}$ tels que $\forall j$, $j$ dans $Z$, $j<switchDone$ on a $x_{j}=tab_{0}[(startindex + (j\times decalage))\ \% tab.length]$ sont déplacés aux index $(startindex + ((j+1)* decalage))\%tab.length$\\

\noindent \textbf{Condition d'arrêt:} switchDone == nSwitch\\

\noindent \textbf{Init:}\\
switchDone := 0;\\
i := (startIndex + decalage) \% tab.length;\\
tmp := tab[startIndex];\\

\noindent \textbf{Iter: }\\
tmp := switchUnique(tab, i, tmp);\\
i := (i+decalage) \% tab.length;\\
switchDone := switchDone + 1;\\

\noindent \textbf{Clot: } -\\



\subsubsection*{$\{$Pre$\}$ Init $\{$Inv$\}$}

$\{startIndex = startIndex_{0}, decalage = decalage_{0}, nswitch = nswitch_{0} \}$ tq (startIndex$_{1}$ = startIndex$_{0}$ \\
$\wedge$ decalage$_{1}$ = decalage$_{0}$\\
$\wedge$ nswitch$_{1}$ = nswitch$_{0}$\\
$\wedge$ tab = tab$_{0}$ [val$_{1}^{0}$, val$_{1}^{1}$, ... ,val$_{1}^{tab.length-1}$])\\

INIT \\

$\{switchDone = 0,\ i = (startIndex_0 + decalage_0)\ \% tab.length, tmp = tab_0[startindex_0]\}$ tq (...)\\
on a bien : 
\begin{itemize}
	\item $0\leq switchDone < tab.length$
	\item $i = startindex + (switchDone+1\times decalage)\ \% tab.length $car$ switchDone = 0$
	\item $tmp = tab[(startIndex + (SwitchDone\times decalage))\ \% tab.length]$ car $switchDone = 0$
	\item il n'existe pas de $j \in \mathbb{Z}$, $j$< SwitchDone = 0 donc la dernière conjonction de notre invariant est vraie. \\
\end{itemize}


L'invariant est donc vérifié sur Init.

\subsubsection*{$\{$Inv $\&\&$ !H$\}$ Iter $\{$Inv$\}$}

$\{ switchDone = switchDone_{1}, i = i_{1}, tmp = tmp_{1}, startIndex = startIndex_{1}, decalage = decalage_{1}, nswitch = nswitch_{1}, tab = tab_{1} \}$ tq (startIndex$_{1}$ = startIndex$_{0}$\\
$\wedge$ $0 \leq$ startIndex < tab.length \\
$\wedge$ decalage$_{1}$ = decalage$_{0}$\\
$\wedge$ nswitch$_{1}$ = nswitch$_{0}$\\
$\wedge$ $0\leq switchDone_{1} \leq nSwitch_{1} < tab.length$\\
$\wedge$ $0\leq i_{1} < tab.length$\\
$\wedge$ $i_{1} = startIndex_{1} + (switchDone_{1}+1\times decalage_{1}) \% tab.length$\\
$\wedge$ $tmp_{1} = tab[(startIndex_{1} + (switchDone_{1}\times decalage_{1}))  \% tab.length]$\\
$\wedge$ Les éléments x$_{j}$ tels que $\forall j$, $j$ dans $Z$, $j<switchDone_{1}$ on a $x_{j}=tab_{0}[(startindex_{1} + (j\times decalage_{1}))\ \% tab.length]$ sont déplacés aux index $(startindex_{1} + ((j+1)* decalage_{1}))\%tab.length$\\

La fonction $switchUnique$ peut être appelé car les pré-conditions sont biens respectées. En effet, tab est bien un tableau d'entiers, tab est non null et $i$ est un indice valide du tableau. \\

\begin{center}
$tmp = switchUnique(tab, i, tmp);$\\
\end{center}

Donc $tmp = tab_{1}[i_1]$, ce qui correspond à $tmp = tab_{1}[startindex_{1} + ((switchDone_{1}+1)\times decalage_{1}) \% tab.length]$.\\

Le tableau a été modifié : $tab = tab_{1}[val_{1}^{0},val_{1}^{1}, ..., val_{1}^{(i_{1}-1)}, tmp_{1}, ...,val_{1}^{tab.length-1}]$.\\

\begin{center}
  $i = (i+decalage) \% tab.length;$\\
 \end{center} 
 
 Dans l'environnement : $i = i_{1} + decalage \% tab.length = startindex_{1} + ((switchDone_{1}+1)\times decalage_{1}) \% tab.length  + decalage_{1} \%tab.length = startIndex_{1} + ((swhitchDone_{1}+2)\times decalage_{1}) \%tab.length$.\\
 
\begin{center}
 $switchDone++;$\\
\end{center}
 
Après l'ITER, l'environnement devient donc : \\

\noindent$\{switchDone = switchDone_{1}+1,\ i = startIndex_{1} + ((swhitchDone_{1}+2)\times decalage_1) \%\ tab.length,$\\
$\ tmp = tab_{1}[startindex_{1}$ $+ ((switchDone_{1}+1)\times decalage_{1}) \% tab.length],\ startIndex = startIndex_{1}, decalage = decalage_{1},\ nswitch = nswitch_{1},\ tab = tab_{1}[val_{1}^{0},val_{1}^{1}, ..., val_{1}^{(i_{1}-1)}, tmp_{1}, ...,val_{1}^{tab.length-1}]\}$\\
tq (...).\\

On obtient bien notre invariant:
\begin{itemize}
	\item comme switchDone est différent de nSwitch et inférieur ou égal, on sait qu'il est toujours inférieur à tab.length
	\item $tmp = tab_{1}[startindex_{1} + ((switchDone_{1}+1)\times decalage_{1}) \% tab.length]$ 
		\subitem $= tab_{1}[startindex + (switchDone\times decalage) \% tab.length]$
	\item $i = startIndex_{1} + ((swhitchDone_{1}+2)\times decalage_{1}) \% tab.length $
		\subitem $= startIndex + ((swhitchDone+1)\times decalage) \%tab.length$
	\item pour $tab$, on a $tab = tab_{1}[val_{1}^{0},val_{1}^{1}, ..., val_{1}^{((i_{1})-1)}, tmp_{1}, ...]$, avec $tmp_{1} = tab_{1}[(startIndex_{1} + (SwitchDone_{1}\times decalage_{1})) \% tab.length]$, et donc avec le nouvel environnement : $tmp_{1} = tab_{1}[(startIndex + ((switchDone-1)\times decalage)) \% tab.length].$ 
	\item Grâce à notre invariant nous savons déjà que les éléments $x_{j}$ tels que $\forall j$, $j \in \mathbb{Z}$, $j<switchDone_{1}$ on a $x_{j}=tab_{0}[(startindex_{1} + (j\times decalage_{1})) \% tab.length]$ sont déplacés aux index $(startindex_{1} + ((j+1)\times decalage_{1}))\% tab.length$. On a maintenant $\forall j, j$ dans $\mathbb{Z}$ et $j<switchDone=switchDone_{1} + 1$, donc seul l'élément $x_{j}$ avec $j= switchDone - 1 = switchDone_{1}$ a du être déplacé. Et c'est le cas, il s'agit de l'élément qui était dans $tmp_{1}$ et qui a été placé à l'index :$ startindex_{1} + ((switchDone_{1}+1)\times decalage_{1}) \% tab.length$. Grâce à notre théorie du problème nous savons que cet élément n'est pas un élément qui avait déjà été déplacé. Ceci nous donne donc la dernière conjonction de notre invariant. \\
\end{itemize}

Notre invariant est donc vérifié sur Iter. \\

\subsubsection*{$\{$Inv $\&\&$ H$\}$ Clot $\{$Post$\}$}

$\{switchDone = switchDone_{1}, i = i_{1}, tmp = tmp_{1}, startIndex = startIndex_{1}, decalage = decalage_{1}, nswitch = nswitch_{1}, tab = tab_{1}\}$\\ 
tq $(startIndex_{1} = startIndex_{0}$\\
$\wedge$ $0 \leq startIndex < tab.length$\\
$\wedge$ $decalage_{1} = decalage_{0}$\\
$\wedge$ $nSwitch_{1} = nSwitch_{0}$\\
$\wedge$ $0 \leq switchDone_{1} \leq nSwitch_{1} < tab.length$\\
$\wedge$ $0\leq switchDone_{1} < tab.length$\\
$\wedge$ $0\leq i_{1} < tab.length$\\
$\wedge$ $i_{1} = startindex_{1} + ((switchDone_{1}+1)\times decalage_{1}) \% tab.length$\\
$\wedge$ $tmp_{1} = tab_{1}[(startIndex_{1} + (switchDone_{1}\times decalage_{1}))  \% tab.length]$\\
$\wedge$ les éléments $x_{j}$ tels que $\forall j, j$ dans $\mathbb{Z}, j<switchDone_{1}$ on a $x_{j}=tab_{0}[(startindex_{1} + (j\times decalage_{1})) \% tab.length]$ sont déplacés aux index ($startindex_{1} + ((j+1)\times  decalage_{1}))\% tab.length$\\
$\wedge$ $switchDone_{1} == nSwitch_{1})$ \\

La post condition de notre sous problème demande que : le tableau est modifié de manière à ce que les éléments aux positions $P_{i}$ sont déplacés aux positions $P_{i+1}$ pour tout $i$ appartenant à $[0,nSwitch[$ tel que $P_{x} = (startIndex + (decalage\times x)) \% tab.length$.

Comme H est vrai, nous avons $switchDone == nSwitch$. Donc en remplaçant switchDone par nSwitch dans la dernière conjonction de notre invariant, nous savons que : les éléments $x_{j}$ tels que $\forall j, j$ dans $\mathbb{Z}, j<nSwitch$ on a $x_{j}=tab_{0}[(startindex + (j\times decalage)) \% tab.length]$ sont déplacés aux index $(startindex + ((j+1)\times decalage)) \% tab.length.$ 
Cela correspond exactement à notre postcondition.\\

Nous avons donc vérifié le tripler d'Hoar-Mana, ce qui nous montre que cet invariant est correct pour cet algorithme. 
Nous avons donc effectué la correction partielle de ce sous problème. 


\subsubsection*{Preuve de terminaison}
Variant  : $tab.length - switchDone$

Inv $\Rightarrow 0 \leq switchDone \leq tab.length$ donc $tab.length - switchDone \geq 0$. Notre variant est bien positif ou nul.\\ 

Vérifions maintenant que notre variant décroit sur Iter.\\

Prenons Inv $\&\&$ !H \\

E = $\{ switchDone = switchDone_{1}, i = i_{1}, tmp = tmp_{1}, startIndex = startIndex_{1}, decalage = decalage_{1}, nswitch = nswitch_{1}, tab = tab_{1}\}$\\
tq $(startIndex_{1} = startIndex_{0}$
$\wedge$ $0 \leq startIndex < tab.length$\\
$\wedge$ $decalage_{1} = decalage_{0}$\\
$\wedge$ $nSwitch_{1} = nSwitch_{0}$\\
$\wedge$ $0 \leq switchDone_{1} \leq nSwitch_{1} < tab.length$\\ 
$\wedge$ $0 \leq switchDone_{1} < tab.length$\\
$\wedge$ $0 \leq i_{1} < tab.length$\\
$\wedge$ $i_{1} = startindex_{1} + ((switchDone_{1}+1)\times decalage_{1}) \% tab.length$
$\wedge$ $tmp_{1} = tab_{1}[(startIndex_{1} + (switchDone_{1}\times decalage_{1})) \% tab.length]$
$\wedge$ les éléments $x_{j}$ tels que $\forall j, j$ dans $\mathbb{Z}$, $j<switchDone_{1}$ on a $x_{j}=tab_{0}[(startindex_{1} + (j\times decalage_{1})) \% tab.length]$ sont déplacés aux index $(startindex_{1} + ((j+1)\times decalage_{1})) \% tab.length$.
$\wedge$ $switchDone_{1} != nSwitch_{1})$.\\

La seule instruction intéressante est : switchDone++.\\
Cela nous donne un nouvel environnement :\\

E' = $\{switchDone = switchDone_{1}+1, ...\}$ tq (...).\\

$Var(E) = tab.length - switchDone_{1}$
$Var(E') = tab.length - switchDone = tab.length - (switchDone_{1}+1)$

On a donc : $Var(E) = tab.length - switchDone_{1} > tab.length - (switchDone_1 +1) = tab.length - switchDone = Var(E')$.\\

On a donc bien un variant qui décroit sur Iter.\\

Ceci nous montre que ce sous problème ce finira toujours. 

Nous avons donc fini la correction partielle et prouvé que cet algorithme s'arrête toujours. On peut donc en conclure que cet algorithme est correct, suivant ces spécifications. \\




