Invariant :

\begin{tabular}{ll}

& $\mathrm{decalage}=\mathrm{decalage}_{0}$ \\
$\bigwedge$ & $\forall i \in \mathbb{Z} \mid i < \mathrm{switchDone} $ \\
$\bigwedge$ & $\forall j \in \mathbb{Z} \mid j <
\frac{\mathrm{tab.length}}{\mathrm(pgcd}}$ \\
$\Rightarrow$ & les éléments aux indices $tab_{0}[(j+i) \% tab.length]$
sont déplacés aux indices $((j*decalage)+i) \% tab.length)$
\end{tabular}

\{Pre\} Init \{Inv\}

$\{decalage=decalage_{0}, pgcd =\cdots, switchDone=0$
$pgcd_{1} = \cdots$

On a bien qu'il n'existe aucun $i < \mathrm{switchDone}$ et donc
qu'aucun élément n'est déplacé au sein de tab.

L'invariant est donc vérifié sur INIT.

\{Inv && !H\} Iter \{Inv}

A chaque itération, switchCadran2 est appelé avec les arguments : tab,
switchDone (incrémenté), decalage et tab.length / pgcd. De ce fait,
selon la spécification de switchCadran2 prouvé précédemment, nous avons
que chaque élément $x \in [0\cdots
    \frac{\mathrm{tab.length}}{\mathrm{pgcd}}[$ sont déplacés à la
        position $(x*\mathrm{decal})+i \% \mathrm{tab.length}$ avec $i <
        \mathrm{switchDone}$

On a donc que les éléments aux indices
$tab_{0}[(j+i)\%\mathrm{tab.length}]$ sont déplacés aux indices
$((j*decalage) + i) \% \mathrm{tab.length}$ pour chaque $i <
\mathrm{switchDone}$
et $j \leq \frac{\mathrm{tab.length}{\mathrm{pgcd}}$

\{Inv \&\& H\} Clot \{Post\} Nous avons que switchDone=pgcd ce qui
implique qu'on a déplacé
$\mathrm{pgcd}*\frac{\mathrm{tab.length}}{\mathrm{pgcd}}$ éléments et
que chacun d'entre eux est déplacé une et une seule fois (comme démontré
plus haut). Tous les élémetnts de tab sont donc déplacés de décal positions.
